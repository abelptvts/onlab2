\chapter{Decentralizált signaling}\label{ch:decentralizált-signaling}


\section{Elvárások}\label{sec:elvarasok}

A signaling szolgáltatás helyettesítésének érdekében ez az egység ennek az elvárásait tárgyalja.

\paragraph{Felelősség}

A signaling szolgáltatás fő felelőssége az ICE candidate csomagok fogadása és irányított (vagy adott esetben broadcast-olt)
továbbítása a többi fél felé.
Ahhoz, hogy a szolgáltatás eljuttasson egy csomagot a célhoz, egy \emph{kétirányú} kommunikációs csatornára van szüksége.

\paragraph{Csomagok mérete}

A csomagok ICE candidate-ek, amik szöveges adatként vannak reprezentálva.
Egy ilyen csomag alacsony szintű hálózati adatokból áll, a következő bejegyzés egy példa erre:
\begin{lstlisting}[label={lst:lstlisting}]
    a=candidate:1 1 UDP 2130706431 192.168.1.102 1816 typ host
\end{lstlisting}
Ahogy a fenti példa mutatja, az ilyen csomagok mérete elenyésző.

\paragraph{Csomagok számossága}

Mivel a signaling folyamat használatára csupán a WebRTC kapcsolat felépítése és esetleg újraépítése során kerül sor, a
csomagok várható számának aránya a kommunikációban résztvevő felek számával meglehetősen alacsony.

\paragraph{Átviteli sebesség}

Az átviteli sebesség nagy mértékben függ attól, hogy mire és milyen körülmények között alkalmazza egy adott rendszer
a WebRTC protokollt.
A dolgozat keretein belül 10 másodperc alatti átlagos válaszidőt várunk el a signaling során.


\section{Elosztott hash-táblák}\label{sec:elosztott-hash-táblák}

Az elosztott hash-táblák olyan elosztott rendszerek, amelyek kulcs-érték párok decentralizált tárolására biztosítanak
lehetőséget.
Az ilyen rendszerekben résztvevő csomópontok csupán a teljes adathalmaz egy bizonyos részét tárolják, nem kizárólagos
módon, tehát az adathalmaz egy adott részét több csomópont is tárolhatja.
Lekérdezés során amennyiben egy adott csomópont nem tárol egy adott kulcs-érték párt, lekérdezi ezt a hozzá fizikailag, vagy
valamilyen más metrika (például az egyedi azonosítók közti különbség) szerint közel levő csomópontoktól.

A lekérdezések a DHT felépítésétől függően több algoritmus alkalmazásával megvalósíthatók, ilyenek például a Koorde,
Kademlia vagy a Tapestry.
Kademlia használata esetén egy lekérdezés komplexitása $\mathcal{O}(\log{}n)$, ahol $n$ a résztvevő csomópontok száma.
A felépítésük és az alapvető koncepcióik miatt a DHT-k extrém méretekig skálázhatóak.

A BEP44\cite{BEP} protokoll lehetővé teszi azt, hogy az elosztott hash-táblákat ne csak kulcs-érték párok tárolására használhassuk,
hanem tetszőleges adatot elérhetővé tudjunk tenni ezek segítségével.
A protokoll használata esetén a változhatatlan adatok kulcsa maga az adat SHA1 hash-je lesz, mivel ilyen esetben nincs
szükség a tartalom aláírására és az író fél azonosítására — a lekérdező fél tudja, hogy mit szeretne megkapni.
Változékony adatok esetében, viszont az az elvárt működés, hogy a kulcs módosítása nélkül változhasson az adat.
Ebben az esetben az adat kulcsa az ító fél publikus kulcsának és egy opcionális salt érték konkatenációjából tevődik
össze, így egy csomópont több kulcs alá is publikálhat értékeket és az olvasó feleknek elég csak az író publikus kulcsát
ismerniük a lekérdezéshez.
Egy írási művelet az író fél privát kulcsával van aláírva.

A konzisztencia megőrzésének érdekében a csomópontok kötelesek egy szekvencia számot megadni megadni minden íráskor és
módosításkor.
Ez a szám monoton nő minden frissítéskor és szerepel az aláírásban.
Adatok írása során a csomópontok csak akkor tárolják a módosításokat, ha a lokális szekvencia számuk kisebb mint az íráskor
megadott.
Lekérdezés során az olvasó opcionálisan megadhatja az adat utolsó ismert szekvencia számát.
Ebben az esetben a többi csomópont csak akkor ad vissza adatok, ha saját szekvencia számuk nagyobb vagy egyenlő az olvasóénál.
A következő bejegyzés változékony adatok írását mutatja be:

\begin{lstlisting}[label={lst:lstlisting2}]
{
    "a": {
        "cas": <optional expected seq-nr (int)>,
        "id": <20 byte id of sending node (string)>,
        "k": <ed25519 public key (32 bytes string)>,
        "salt": <optional salt to be appended to "k" when hashing (string)>
        "seq": <monotonically increasing sequence number (integer)>,
        "sig": <ed25519 signature (64 bytes string)>,
        "token": <write-token (string)>,
        "v": <any bencoded type, whose encoded size < 1000>
    },
    "t": <transaction-id (string)>,
    "y": "q",
    "q": "put"
}
\end{lstlisting}

A decentralizált signaling szempontjából a BEP44 protokoll legfőbb hátrányai a következők:
\begin{itemize}
    \item Egy adott kulcs alatt tárolt adat maximális mérete 1000 byte.
    \item Egy adott kulcs alatt tárolt adatot csak az a csomópont módosíthat, amelyik eredetileg elérhetővé tette.
\end{itemize}

A korlátolt méret hátránya megkerülésére többféle eljárás létezik, például az adatok csonkolása és láncolt lista alkalmazása.

A bemutatott BEP44 protokoll lehetőséget ad WebRTC signaling megvalósítására centralizált köztes
fél bevezetése nélkül.


\section{Két résztvevő közti WebRTC csatorna felépítése}\label{sec:két-peer-közti-webrtc-csatorna-felépítése}

Kezdetben mindkét résztvevő csatlakozik egy publikus tracker-ökből\footnote{Olyan publikusan elérhető számítógépek,
    amelyek egy olyan DHT-t alkotnak, amiben bármit tudunk ideiglenesen tárolni.} álló elosztott hash-táblába.
Mindkét csomópont ismeri a másik publikus kulcsát, ezáltal hozzáférnek bármihez, amit a másik publikál.

A csomópontok két-két \emph{kupacba} (továbbiakban bucket) írnak adatot: \emph{offers}, ami a csatlakozáshoz szükséges
WebRTC offer-eket tartalmazza, és \emph{ICE candidates}, ami a kapcsolat felépítéséhez szükség ICE candidate-eket tartalmazza.
Mindkét csomópont időnként lekérdezi a másik két bucket-jének tartalmát a publikus tracker-ökön keresztül, ezáltal
értesülve az esetleges hálózati változásokról.

Az egyik fél kezdeményezi a kapcsolat felépítését egy kezdeti WebRTC offer létrehozásával és tárolásával.
A másik fél ezt észleli, létrehoz egy választ az ajánlatra és elkezdődik az ICE candidate-ek cseréje.
Miután ez megtörtént, a kezdeményező fél nyit egy WebRTC datachannel-t (RFC 8831\cite{RFC_8831}), erről a másik (már WebRTC-n keresztül) értesül.
Ezzel létrejön egy duplex csatorna a két fél között.
